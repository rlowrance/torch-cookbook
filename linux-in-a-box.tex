% vim: set textwidth=72 :
% Copyright (C) 2013 Roy E Lowrance
% See the file COPYING.txt for copying conditions.

\documentclass{article}
\usepackage{amssymb,latexsym,amsmath}
\usepackage{upquote}
\let\code\texttt % in line source code

\begin{document}
\title{Linux in a Box}
\author{Roy E Lowrance}
\maketitle

%\abstract{XXX}

\section{Problem}
You want to intall Ubuntu 12.04 in a virtual machine. You may want to do
this so that you can continue to run the operating system that came with
your computer or so that you can install use software that supports
Ubuntu (such as Torch).

\section{Solution}

\subsection{Introduction}

The solution is to install VirtualBox and run Ubuntu 12.04 LTS as a
guest inside of VirtualBox. Getting everything to work is tricky. These
directions are meant to work, but have not been tested on all hardware
and software combinations.

\subsection{Download iso's}

Download the correct VirtualBox installer for your operating system. Go
to \code{www.virtualbox.org}, select Downloads (in the left panel), pick
the right platform package for your operating system. If your operating
system is recent, it is a 64-bit operating system. So pick a 64-bit
version of VirtualBox. I used VirtualBox 4.2.18 for Linux in preparing
this note.

You should navigate to the online VirtualBox documentation on the
VirtualBox web site and bookmark it. You will need to refer to it in
case these direction do not work on your system. There is a lot you can
do with VirtualBox that is not covered in this note, and the
documentation is pretty clear once you understand the concepts.

Download the iso for ubunutu 12.04.03.  Go to \code{www.ubuntu.com},
select download \textless desktop, pick 64-bit if your operating system is not
32 bit (your operating system is most likely 64-bit if your system is
only a few years old), and select Ubuntu 12.04 LTS. Canonical, the
developer of Ubuntu, promises support for this release of Ubuntu until
April 2017. Very seldom will upgrades to operating systems make your
life as a data scientist any better, so its better to put off OS updates
until you are forced.

\subsection{Install Virtual Box}

Follow your operating system's procedure to install Virtual Box.

If your OS in Ubuntu, you will need to open a terminal and execute these
command to add your user name to the virtual box users group.

\begin{verbatim}
$ sudo groupadd vboxusers
$ sudo usermod -aG vboxusers <your user name	extgreater
\end{verbatim}

\subsection{Create a virtual machine}

A virtual machine is a simulated computer. It is called a ``guest.'' The
guest runs on the ``host,'' the physical computer.

\begin{itemize}
  \item Start Virtual Box
  \item Click on ``New'' in the GUI to create a new virtual machine.
  \item Click ``Next.''
  \item Name: Set the virtual machine name to "ubuntu-12.04.03-64-bit"; 
  \item Memory: accept the default 512 MB
  \item Hard drive: create a virtual hard drive of type VDI, dynamically
    allocated, 16 GB in size.
\end{itemize}

\subsection{Install Ubuntu on the virtual machine}

Put the virtual CD ROM (the iso file) with the operating system into the virtual CD
drive of the virtual machine you just created.

\begin{itemize}
  \item In the VirtualBox GUI, make sure the virtual machine is
    selected.
  \item Select the Settings icon on the top panel.
  \item Select Storage from the left panel.
  \item Under IDE Controller, click on Empty (the icon for the CD
    drive).
  \item In the right panel, click attribute Live CD/DVD.
  \item In the right panel, click the CD icon, select choose a virtual
    CD/DVD disk file, navigate to your Ubuntu iso file, select it, and
    click OK.
\end{itemize}

Boot and configure Ubuntu in the guest virtual machine.
\begin{itemize}
\item In the VirtualBox GUI, select the virtual machine and click
  Start. VirtualBox will start the virtual machine. You will see the
  Ubuntu boot message. Ignore any error messages.
\item You may be prompted to pick a language. Choose English for now so
  that other people can help you. Later you can switch to your favorite
  language.
\item Select Install Ubuntu.
\item Preparing to install Ubuntu. Don't select Download updates while
  installing (unless you have a lot Internet bandwidth). Select
  Install third-party software. Select Continue.
\item Installation Type. Select Erase disk and install Ubuntu. You are
  erasing the virtual disk you created when you made the guest machine,
  not your host system disk. Continue.
\item Erase disk and install Ubuntu. Install Now.
\item Where are you? Select New York; Continue. This sets the time zone
  for the virtual computer.
\item Keyboard layout. Find your keyboard and select it. You want the
  keyboard that is on your physical system. Continue.
\item Who are you? Enter your name, pick a user name, password. Select
  require password to log in. Do not encrypt your home folder. Continue.
\item Watch the advertising. Don't move your mouse outside of the
  virtual screen, in an attempt to avoid a bug in VirtualBox.
\end{itemize}

When the installation is finished, click ``restart now.'' If the guest
freezes, shut it down by clicking on the close button in your GUI and
selecting ``power off machine.'' In the VirtualBox window, select the
virtual machine and settings. Remove the virtual CD from the guest:
Settings 	\textgreater Storage \textgreater select the virtual CD
\textgreater select the icon for the
virtual CD \textgreater click ``remove from virtual drive.'' Now restart the
virtual machine.

If the guest did not freeze,
you will be told to remove the installation media
(which is the virtual CD, the iso file) and press Enter. You can remove
the virtual CD by right-clicking on the CD icon on the bottom panel of the 
VirtualBox GUI panel. After the guest system reboots, shut it down. The Ubuntu shutdown
command is under the system menu in the upper right hand corner.

\subsection{Hint on keyboard and mouse capture}

VirtualBox tried to figure out whether mouse movement and clicks and
keyboard keystrokes should be sent to the host or to the guest.
Sometimes it gets confused and the input focus gets stuck in
the guest. When that happens, you need to tell the guest
to release input focus. Look at the VirtualBox GUI. The lower
right hand corner is the key combination that releases focus from the
virtual machine.

\subsection{Configure the virtual machine}

Shut down the virtual machine if you have not already done so.

You are in the VirtualBox GUI. Select the Ubuntu virtual machine and
click settings.  On the System tab, set the Base Memory to something
higher than the default. I set it to the max that is in the green area.
That's because I tend to do most of my work in the guest OS, not the
host OS, so I want the guest to have as much memory as possible. If you
use your host system a lot, you will want to make your guest smaller
than your host. Note that you can change this setting easily as your
usage of your system changes.

In the Shared Folders tab, add a shared folder that is your home
directory in the host system. Choose full access and auto-mount.

Restart the guest system. Log in.

Install the dkms file system in Ubuntu. Open a terminal and type
\code{sudo apt-get install dkms}.

In the VirtualBox menu, select Devices \textgreater Install Guest
Additions. Click Run when asked if you trust the software. When the
installation finished, press Return as instructed.

Resize the window holding the display for the guest. The Ubuntu display
should resize to fit the window. If it does not, shut down the virtual
machine and restart it. 

In the VirtualBox menus, pick View \textgreater Full Screen.  Read the
message and memorize the Host key. Then click Switch.  Mouse down to the
bottom of the screen. You should see a tiny strip.  That's a menu bar
that will pop up. Exit full-screen mode by picking the multiple-window
icon.

Remove the guest additions virtual CD by using the VirtualBox Device
menu.

\subsection{Configure and install Ubuntu software}

Your vitual machine should be running Ubuntu.

Install synaptic, a software package manager. Click on the icon for the
Ubuntu software center in the left strip. Search for ``synaptic'' and
select ``Synaptic Package Manager.'' Install it. Once it installs, exit
the Ubuntu Sofware Center.

Start synaptic by searching for it using the dash home. Ubuntu comes
with \code{gedit}, an easy-to-use GUI-requiring text editor. You will want to use a
text editor that does not require a GUI. Their are two equally good
choices: emacs and vim. Emacs is easier to use for beginners and some
say that vim is faster to use for experts. You should pick one and use
it, as going back and forth is just too confusing. I recommend using
tmux (to multiplex terminal sessions) and tmux is a bit more friendly to
vim than emacs. So I use vim. But before switching to vim, I had happily
used emacs for many years.

Use the Quick Filter to find emacs or vim and mark it for installation.
Choose Apply (under the check mark) to install all the software you
picked. Synaptic will download the software and install it. Later, if
the software gets updated, Synaptic will prompt you to install the
updates.

In the guest, select the folder icon in the strip on the left. You should
see your home folder. Check the devices section of the panel. If it has
a CD-ROM containing the Virtual Box Additions, click on the eject icon
to its right.

Now let's gain access to your home folder in the host. Previously you
configured your guest machine to have full access to your home folder.
Now you need to give your Ubuntu user id permissio to access this
folder. You do that by adding yourself to the vboxsf user group. (A user
may be a member of many groups. Being a member of a group gives a user
permission to read, write, and execute certain files.) To add yourself
to the vboxsf group, open a terminal and execute:

\begin{verbatim}
$ sudo groupadd vboxsf
$ sudo usermod -aG vboxsf <your user name>
\end{verbatim}

Don't worry if the first command fails.


Log out and log back in in order to active your membership in the new
group. The icon in the upper right corner of the
Ubuntu screen is a drop-down menu. One menu pick will log you out.

Now let's access the home folder in the host system. Reopen the file
browser. Click on file system. Select the media folder. In it, you
should see a folder whose name begins with ``fs\_''. This is your home
folder in the host. It is visible in the guest. Double click on this
folder to open it. 

You should update the operating system software. Go to the system menu,
select Updates available and follow the prompts. At the end of the
update, you will need to reboot. Do so, then shut down the guest.



\section{Discussion}

\subsection{Sofware selections}

VirtualBox is specified as the virtual machine manager because it is
free and runs on Linux, OS X, and Windows. Other virtual managers can be
used. Search for virtualbox alternatives.

I have found VirtualBox to be difficult to work with. It does not always behave
consistently. Sometimes to get it to work, I have to restart the guest,
VirtualBox, and/or my host  system. Once I misconfigured VirtualBox and its
locked up my Windows 7 system. If the GUI freezes, try not moving input
focus away from the virtual machine's window after you restart.

The version of linux recommended is Ubuntu 12.04 because it is popular
and torch is tested on it. Because it is popular, there is a lot of
software that is preconfigured for it and a lot of help available on the
Internet and from colleagues. 

Emacs and vim are suggested as installs because both are
well-established free text editors that work in both a terminal and with
a GUI. You should pick one and learn it.


\subsection{Going further}

The \code{tmux} program lets you run multiple terminal sessions inside
on terminal. It also allows you to reconnect to a server if the
connection is lost. A lost connection includes shutting down your
laptop.

If you have two folders of data files you need to keep in sync--say one
on a server and one on your laptop--\code{rsync} will prove useful.

If you have two folders of program code you need to keep in sync or if
you want to be able to go back to an earlier version of a program you
are developing or if you want to share code development with team
members, learn about \code{git} and GitHub. 

In Ubuntu, you can hide the icons on the left to gain more screen space.
You can use the alt key or Windows key to bring up a panel where you can
search for programs and files. You can add indicators to the top line of
the window including one that monitors CPU usage and memory usage.

In VirtualBox, you can configure a guest to use as many CPUs as are in
the host system.


\subsection{Configuring systems}

I find it convenient to work in one operating system as much as
possible. That operating system should be a linux variant, as linux is
most popular on compuatation servers.

I configure my systems in one of two ways. When I order a system with a
non-linux operating system, I typically install VirtualBox and run a
Ubuntu guest. Sometimes I wipe out the vendor's operating system and
install Ubuntu directly on the hardware. This is usually easier to do
than configuring VirtualBox but does not make the original OS available.
Usually Ubuntu supports all the hardware in the system, but sometimes
not. For that reason, I consider buying systems from vendors that offer
Ubuntu. These vendors typically have up-to-date desktop systems but
their laptop systems tend to lag behind in hardware features.   

I move between systems regularly. For example, I have a desktop and
laptop at home and a desktop at NYU. I use Dropbox to syncronize files
among my systems. In Dropbox, I create a folder for each project. I
consider a course to be a project. The Dropbox folder is in my home
folder on each system. Sometimes I access the files for a project in the
host OS, sometimes in the Ubuntu guess. I do all the programming work in
Ubuntu.

Dropbox doesn't work on NYU's servers. To keep source code synchronized,
I use github as a central point and push and pull to my github account.
I keep one github repository for each project I am working on. If data
files are small and not changing frequently, I put them on github. If
they are large or frequently changing, I use rsync to synchronize them
among systems.

\section{See also}

There is a lot of documentation and tutorial on both emacs and vim on
the Internet.

Oracle's VirtualBox web site contains the definite documentation on
VirtualBox.

This recipe is free documentation. You can modify it by visiting the
github for account ``rlowrance'' and forking the repo
``torch-cookbook.''

\end{document}

