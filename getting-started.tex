% vim: set textwidth=72 :
% Copyright (C) 2013 Roy E Lowrance
% See the file COPYING.txt for copying information

\documentclass{article}
\usepackage{amssymb,latexsym,amsmath}
\let\code\texttt % in line source code

\begin{document}
\title{Getting Started}
\author{Roy E Lowrance}
\maketitle

%\abstract{XXX}

\section{Problem}
You need to know the basic: how to start torch, stop torch, and get
help.

\section{Solution}

To learn how to start and stop torch and the basics for using it, follow
the tutorial at \code{www.torch.ch/manual/tutorial/index}.

The main forum for torch7  is at 
\code{https://groups.google.com/forum/\#!forum/torch7}. Use it to post
questions and provide answers to others.

Torch is open source and hosted at GitHub as project \code{torch7} for
user \code{torch}. The URL is \code{https://github.com/torch/torch7}. 

\section{Description}

Torch is a library for Lua.

\code{www.lua.com} descibes lua as
\begin{itemize}
  \item fast: faster than other scripting languages. There is a
    just-in-time compiler available that makes lua programs even faster.
    The standard install for torch installs the just-in-time compiler.
  \item portable: Lua is written in standard C is is hence highly
    portable. You can run it on large systems and very small systems.
  \item embeddable: You can write a program in a language that can call
    C functions and embed the entire lua language into your program.
    This feature has made lua popular as a scripting engine for
    applications including games.
  \item powerful: lua provides a few facilities that allow
    for object-oriented programming in a user-defined way.
  \item small: the source code and Lua library take 243K bytes.
  \item free: distributed under the MIT license.
\end{itemize}

In addition to these attributes, lua is simpler than some other
scripting languages. It provides one data structure, the \code{table}
where some languages provide both lists and hash tables (dictionaries).
It has only one numeric data type, the floating point double in torch's
instantiation of lua, instead of integers, reals, and complex numbers.

Moreover, lua has a simple syntax that is familiar to C programmers.

Torch extends lua:
\begin{itemize}
  \item Defines a class structure. Lua itself provides only a way to
    create classes and objects. Torch implements a class and object
    facility.
  \item Defines a tensor structure. A tensor is a compactly-stored
    n-dimensional array
    of numbers. Up to 8 dimensions are supported.
  \item Provides a function library that mimics many MATLAB functions. 
\end{itemize}


Torch might be preferred to MATLAB when speed, avoidance of software
licensing costs, and ability to operate on small devices is a concern
for a project. If only the licensing cost is a concern, \code{octave} is
a free implemenation of the basic components of MATLAB.

Compared to MATLAB programs, torch programs can be better structured,
thanks to careful thinking of the lua design team. However, torch
programs tend to be longer than MATLAB programs, because there is no
special syntax in lua to directly articulate matrix operations and
matrix literal values.




\section{See also}

A description of torch is this paper:\\
R. Collebert, K. Kavukcuoglu, and C. Farabet. \emph{Torch7: A
Matlab-like Environment for Machine Learning}. in \emph{BigLearn, NIPS
Workshop, 2011}.

This paper claims that Torch7 provides a flexible programming
environment with good performance:\\
Roman Collobert,Koray Kavukcuoglu, and Clement Farabet.
\emph{Implementing Neural Networks Efficiently}. In \emph{Neural
  Networks: Tricks of the Trade, G. Montavon, G. Orr, and K-R Muller
(Ed), Springer. 2012)}.

Both papers may be downloaded from the publications section of 
Collobert's website\\
\code{ronan.collobert.com}



\end{document}


